%% For double-blind review submission, w/o CCS and ACM Reference (max submission space)
\documentclass[sigplan,screen]{acmart}
\settopmatter{printfolios=true,printccs=false,printacmref=false}
%% For double-blind review submission, w/ CCS and ACM Reference
%\documentclass[sigplan,review,anonymous]{acmart}\settopmatter{printfolios=true}
%% For single-blind review submission, w/o CCS and ACM Reference (max submission space)
%\documentclass[sigplan,review]{acmart}\settopmatter{printfolios=true,printccs=false,printacmref=false}
%% For single-blind review submission, w/ CCS and ACM Reference
%\documentclass[sigplan,review]{acmart}\settopmatter{printfolios=true}
%% For final camera-ready submission, w/ required CCS and ACM Reference
%\documentclass[sigplan]{acmart}\settopmatter{}

%include polycode.fmt

%%% If you see 'ACMUNKNOWN' in the 'setcopyright' statement below,
%%% please first submit your publishing-rights agreement with ACM (follow link on submission page).
%%% Then please update our instructions page and copy-and-paste the NEW commands into your article.
%%% Please contact us in case of questions; allow up to 10 min for the system to propagate the information.
%%%
%%% The following is specific to META '19 and the paper
%%% 'From Definitional Interpreter to Symbolic Executor'
%%% by Adrian D. Mensing, Hendrik van Antwerpen, Casper Bach Poulsen, and Eelco Visser.
%%%
\setcopyright{rightsretained}
\acmPrice{}
\acmDOI{10.1145/3358502.3361269}
\acmYear{2019}
\copyrightyear{2019}
\acmISBN{978-1-4503-6985-5/19/10}
\acmConference[META '19]{Proceedings of the 4th ACM SIGPLAN International Workshop on Meta-Programming Techniques and Reflection}{October 20, 2019}{Athens, Greece}
\acmBooktitle{Proceedings of the 4th ACM SIGPLAN International Workshop on Meta-Programming Techniques and Reflection (META '19), October 20, 2019, Athens, Greece}

%% Copyright information
%% Supplied to authors (based on authors' rights management selection;
%% see authors.acm.org) by publisher for camera-ready submission;
%% use 'none' for review submission.
\setcopyright{none}
%\setcopyright{acmcopyright}
%\setcopyright{acmlicensed}
%\setcopyright{rightsretained}
%\copyrightyear{2018}           %% If different from \acmYear

%% Bibliography style
\bibliographystyle{ACM-Reference-Format}
%% Citation style
%\citestyle{acmauthoryear}  %% For author/year citations
%\citestyle{acmnumeric}     %% For numeric citations
%\setcitestyle{nosort}      %% With 'acmnumeric', to disable automatic
                            %% sorting of references within a single citation;
                            %% e.g., \cite{Smith99,Carpenter05,Baker12}
                            %% rendered as [14,5,2] rather than [2,5,14].
%\setcitesyle{nocompress}   %% With 'acmnumeric', to disable automatic
                            %% compression of sequential references within a
                            %% single citation;
                            %% e.g., \cite{Baker12,Baker14,Baker16}
                            %% rendered as [2,3,4] rather than [2-4].


%%%%%%%%%%%%%%%%%%%%%%%%%%%%%%%%%%%%%%%%%%%%%%%%%%%%%%%%%%%%%%%%%%%%%%
%% Note: Authors migrating a paper from traditional SIGPLAN
%% proceedings format to PACMPL format must update the
%% '\documentclass' and topmatter commands above; see
%% 'acmart-pacmpl-template.tex'.
%%%%%%%%%%%%%%%%%%%%%%%%%%%%%%%%%%%%%%%%%%%%%%%%%%%%%%%%%%%%%%%%%%%%%%


%% Some recommended packages.
\usepackage{booktabs}   %% For formal tables:
                        %% http://ctan.org/pkg/booktabs
\usepackage{subcaption} %% For complex figures with subfigures/subcaptions
                        %% http://ctan.org/pkg/subcaption
\usepackage{cleveref}
\usepackage{boxedminipage2e}

\crefname{section}{\S}{\S\S}
\crefname{figure}{Fig.}{Fig.}

\begin{document}

%% Title information
\title{From Definitional Interpreter To Symbolic Executor} %% [Short Title] is optional;
                                        %% when present, will be used in
                                        %% header instead of Full Title.
% \titlenote{with title note}             %% \titlenote is optional;
%                                         %% can be repeated if necessary;
%                                         %% contents suppressed with 'anonymous'
% \subtitle{And Its Application to Automating Test Generation for Definitional Interpreters} %% \subtitle is optional
% \subtitlenote{with subtitle note}       %% \subtitlenote is optional;
%                                         %% can be repeated if necessary;
%                                         %% contents suppressed with 'anonymous'


%% Author information
%% Contents and number of authors suppressed with 'anonymous'.
%% Each author should be introduced by \author, followed by
%% \authornote (optional), \orcid (optional), \affiliation, and
%% \email.
%% An author may have multiple affiliations and/or emails; repeat the
%% appropriate command.
%% Many elements are not rendered, but should be provided for metadata
%% extraction tools.

%% Author with single affiliation.
\author{Adrian Mensing}
\affiliation{
  % \position{...}
  % \department{}              %% \department is recommended
  \institution{Delft University of Technology}            %% \institution is required
  % \streetaddress{Street1 Address1}
  % \city{City1}
  % \state{State1}
  % \postcode{Post-Code1}
  \country{Netherlands}                    %% \country is recommended
}
\email{a.d.mensing-1@@student.tudelft.nl}          %% \email is recommended

\author{Hendrik van Antwerpen}
\affiliation{
  % \position{...}
  % \department{}              %% \department is recommended
  \institution{Delft University of Technology}            %% \institution is required
  % \streetaddress{Street1 Address1}
  % \city{City1}
  % \state{State1}
  % \postcode{Post-Code1}
  \country{Netherlands}                    %% \country is recommended
}
\email{h.vanantwerpen@@tudelft.nl}

\author{Casper Bach Poulsen}
\affiliation{
  % \position{...}
  % \department{}              %% \department is recommended
  \institution{Delft University of Technology}            %% \institution is required
  % \streetaddress{Street1 Address1}
  % \city{City1}
  % \state{State1}
  % \postcode{Post-Code1}
  \country{Netherlands}                    %% \country is recommended
}
\email{c.b.poulsen@@tudelft.nl}          %% \email is recommended

\author{Eelco Visser}
\affiliation{
  % \position{...}
  % \department{}              %% \department is recommended
  \institution{Delft University of Technology}            %% \institution is required
  % \streetaddress{Street1 Address1}
  % \city{City1}
  % \state{State1}
  % \postcode{Post-Code1}
  \country{Netherlands}                    %% \country is recommended
}
\email{e.visser@@tudelft.nl}          %% \email is recommended


% \authornote{Delft University of Technology}          %% \authornote is optional;
                                        %% can be repeated if necessary
% \orcid{0000-0003-0622-7639}             %% \orcid is optional
% \affiliation{
%   \position{Assistant Professor}
%   \department{Department1}              %% \department is recommended
%   \institution{Institution1}            %% \institution is required
%   \streetaddress{Street1 Address1}
%   \city{City1}
%   \state{State1}
%   \postcode{Post-Code1}
%   \country{Country1}                    %% \country is recommended
% }

% %% Author with two affiliations and emails.
% \authornote{Delft University of Technology}          %% \authornote is optional;
%                                         %% can be repeated if necessary
% \orcid{nnnn-nnnn-nnnn-nnnn}             %% \orcid is optional
% \affiliation{
%   \position{Position2a}
%   \department{Department2a}             %% \department is recommended
%   \institution{Institution2a}           %% \institution is required
%   \streetaddress{Street2a Address2a}
%   \city{City2a}
%   \state{State2a}
%   \postcode{Post-Code2a}
%   \country{Country2a}                   %% \country is recommended
% }
% \email{first2.last2@@inst2a.com}         %% \email is recommended
% \affiliation{
%   \position{Position2b}
%   \department{Department2b}             %% \department is recommended
%   \institution{Institution2b}           %% \institution is required
%   \streetaddress{Street3b Address2b}
%   \city{City2b}
%   \state{State2b}
%   \postcode{Post-Code2b}
%   \country{Country2b}                   %% \country is recommended
% }
% \email{first2.last2@@inst2b.org}         %% \email is recommended


%% Abstract
%% Note: \begin{abstract}...\end{abstract} environment must come
%% before \maketitle command
\begin{abstract}
Symbolic execution is a technique for automatic software validation and verification.
New symbolic executors regularly appear for both existing and new languages and such symbolic executors are generally manually (re)implemented each time we want to support a new language.
We propose to automatically generate symbolic executors from language definitions, and present a technique for mechanically (but as yet, manually) deriving a symbolic executor from a definitional interpreter.
The idea is that language designers define their language as a monadic definitional interpreter, where the monad of the interpreter defines the meaning of branch points.
Developing a symbolic executor for a language is a matter of changing the monadic interpretation of branch points.
Our long-term goal is to integrate these techniques in language development workbenches and workflows, to make developer-boosting meta-programming techniques such as symbolic execution readily and automatically available to language designers and software developers.
In this paper, we illustrate the technique on a language with recursive functions and pattern matching, and use the derived symbolic executor to automatically generate test cases for definitional interpreters implemented in our defined language.
\end{abstract}


%% 2012 ACM Computing Classification System (CSS) concepts
%% Generate at 'http://dl.acm.org/ccs/ccs.cfm'.
\begin{CCSXML}
<ccs2012>
<concept>
<concept_id>10003752.10010124.10010125.10010129</concept_id>
<concept_desc>Theory of computation~Program schemes</concept_desc>
<concept_significance>500</concept_significance>
</concept>
<concept>
<concept_id>10011007.10010940.10010992.10010998</concept_id>
<concept_desc>Software and its engineering~Formal methods</concept_desc>
<concept_significance>300</concept_significance>
</concept>
<concept>
<concept_id>10011007.10011074.10011092.10011782</concept_id>
<concept_desc>Software and its engineering~Automatic programming</concept_desc>
<concept_significance>100</concept_significance>
</concept>
</ccs2012>
\end{CCSXML}

\ccsdesc[500]{Theory of computation~Program schemes}
\ccsdesc[300]{Software and its engineering~Formal methods}
\ccsdesc[100]{Software and its engineering~Automatic programming}
%% End of generated code


%% Keywords
%% comma separated list
\keywords{Symbolic Execution, Monads, Haskell, Definitional Interpreter}  %% \keywords are mandatory in final camera-ready submission


%% \maketitle
%% Note: \maketitle command must come after title commands, author
%% commands, abstract environment, Computing Classification System
%% environment and commands, and keywords command.
\maketitle

%include forall.fmt

%include sections/Sect01Intro.lhs

%include sections/Sect02SemStd.lhs

%include sections/Sect03TowardsSymExc.lhs

%include sections/Sect04SymExc.lhs

%include sections/Sect05Correctness.lhs

%include sections/Sect06TestGen.lhs

%include sections/Sect07conclusion.lhs

%% Acknowledgments
\begin{acks}                            %% acks environment is optional
                                        %% contents suppressed with 'anonymous'
  %% Commands \grantsponsor{<sponsorID>}{<name>}{<url>} and
  %% \grantnum[<url>]{<sponsorID>}{<number>} should be used to
  %% acknowledge financial support and will be used by metadata
  %% extraction tools.
  We thank the anonymous reviewers for their insightful suggestions for improvements and future research directions, and Sven Keidel for his timely and helpful comments. 
\end{acks}


%% Bibliography
\bibliography{bibfile}

%% Appendix
% \appendix
% \section{Appendix}

% Text of appendix \ldots

\end{document}

%%% Local Variables:
%%% mode: latex
%%% TeX-master: t
%%% End:
